
%%% Template originally created by Karol Kozioł (mail@karol-koziol.net) and modified for ShareLaTeX use

\documentclass[a4paper,11pt]{article}

\usepackage[T1]{fontenc}
\usepackage[utf8]{inputenc}
\usepackage{graphicx}
\usepackage{xcolor}

%\renewcommand\familydefault{\sfdefault}
%\usepackage{tgheros}
%\usepackage[defaultmono]{droidmono}

\usepackage{amsmath,amssymb,amsthm,textcomp}
\usepackage{enumerate}
\usepackage{multicol}
%\usepackage{tikz}
\usepackage{booktabs} 
%\usepackage{geometry}
\usepackage{fancyvrb}
\usepackage{algorithm, algorithmic}
\usepackage[justification=centering]{subfig}
\usepackage[final]{pdfpages}
\usepackage{hyperref}
\usepackage{marginnote}
\usepackage[top=20mm, bottom=20mm, outer=50mm, inner=10mm, heightrounded, marginparwidth=35mm, marginparsep=7mm]{geometry}


\newcommand{\linia}{\rule{\linewidth}{0.5pt}}

% my own titles
\makeatletter
\renewcommand{\maketitle}{
\begin{center}
\vspace{2ex}
{\huge \textsc{\@title}}
\vspace{1ex}
\\
\linia\\
\@author \hfill \@date
\vspace{4ex}
\end{center}
}
\makeatother
%%%

% custom footers and headers
\usepackage{fancyhdr}
\pagestyle{fancy}
\lhead{}
\chead{}
\rhead{}
\lfoot{}
\cfoot{--\thepage--}
\rfoot{}
\renewcommand{\headrulewidth}{0pt}
\renewcommand{\footrulewidth}{0pt}
%

% code listing settings
\usepackage{listings}
\lstset{
    language=Python,
    basicstyle=\ttfamily\small,
    aboveskip={1.0\baselineskip},
    belowskip={1.0\baselineskip},
    columns=fixed,
    extendedchars=true,
    breaklines=true,
    tabsize=4,
    prebreak=\raisebox{0ex}[0ex][0ex]{\ensuremath{\hookleftarrow}},
    frame=lines,
    showtabs=false,
    showspaces=false,
    showstringspaces=false,
    keywordstyle=\color[rgb]{0.627,0.126,0.941},
    commentstyle=\color[rgb]{0.133,0.545,0.133},
    stringstyle=\color[rgb]{01,0,0},
    numbers=left,
    numberstyle=\small,
    stepnumber=1,
    numbersep=10pt,
    captionpos=t,
    escapeinside={\%*}{*)}
}

%%%----------%%%----------%%%----------%%%----------%%%
     
\begin{document}
\title{RPi setup}
\author{Topi}
\date{\today}

\maketitle

\begin{enumerate}
	\item Burn 32 bit buster image using Rpi imager.
	\item Create file ``ssh'' into boot partition.
	\item Create file: wpa\_supplicant.conf into boot partition.
	\begin{lstlisting}
country=SE
ctrl_interface=DIR=/var/run/wpa_supplicant GROUP=netdev
update_config=1

network={
ssid="#Telia-D5B7C8"
psk="*G41H*2r#pE-faR4"
}
\end{lstlisting}
	\item So the wpa did not work with the telia ssid - needed to change that to nkai wifi and then worked\dots What the heck\dots Anyhow, then in \texttt{/etc/wpa\_supplicant/} go and edit the file \texttt{sudo nano wpa\_supplicant.conf} and add the other ssids\dots
	\item \texttt{sudo pat update \&\& sudo apt upgrade}
	\item We try to install \url{https://github.com/mpromonet/v4l2rtspserver.git}
	\begin{itemize}
	\item \texttt{sudo apt install git cmake}
	\item \texttt{git clone https://github.com/mpromonet/v4l2rtspserver.git}
	\item \texttt{cd v4l2rtspserver \&\& sudo cmake . \&\& sudo make \&\& sudo make install}
	\item Now you can run: \texttt{sudo v4l2rtspserver /dev/video0}
	\item On you local machine: open vlc, media -> open network stream -> \texttt{rtsp://<rpi ip>:8554/unicast}
	\item WORKS! Latency around 2sec.
	\item \texttt{sudo raspistill -v -o test.jpg}
	\end{itemize}

\item Installed database \url{https://github.com/andymccurdy/redis-py}: \texttt{sudo apt get redis}, client: \texttt{pip3 install redis}
	\item Redis is a server that is running on RPi. It is running there! Start, by \texttt{redis-server}
	\item Subscribe from rpi to redis server - by python scipt.
	\item On jetson write things into the database/server. 
\end{enumerate}
Enabling servo control using 16ch adafruit controller board.
\begin{enumerate}
	\item create virtual env. \texttt{python3 -m venv venv}
	\item activate it: \texttt{source venv/bin/activate}
	\item  \texttt{sudo apt-get install python-smbus}
	\item \texttt{sudo apt-get install i2c-tools}
	\item Read the i2c devices: \texttt{sudo i2cdetect -y 1}
	\item install circuit python servokit: \texttt{pip3 install adafruit-circuitpython-servokit}
	\item add topiko to i2c group for permissions: \texttt{sudo adduser topiko i2c}
\end{enumerate}<++>
%\bibliography{refs}
%\bibliographystyle{plain}
\end{document}

