
%%% Template originally created by Karol Kozioł (mail@karol-koziol.net) and modified for ShareLaTeX use

\documentclass[a4paper,11pt]{article}

\usepackage[T1]{fontenc}
\usepackage[utf8]{inputenc}
\usepackage{graphicx}
\usepackage{xcolor}

%\renewcommand\familydefault{\sfdefault}
%\usepackage{tgheros}
%\usepackage[defaultmono]{droidmono}

\usepackage{amsmath,amssymb,amsthm,textcomp}
\usepackage{enumerate}
\usepackage{multicol}
%\usepackage{tikz}
\usepackage{booktabs} 
%\usepackage{geometry}
\usepackage{fancyvrb}
\usepackage{algorithm, algorithmic}
\usepackage[justification=centering]{subfig}
\usepackage[final]{pdfpages}
\usepackage{hyperref}
\usepackage{marginnote}
\usepackage[top=20mm, bottom=20mm, outer=50mm, inner=10mm, heightrounded, marginparwidth=35mm, marginparsep=7mm]{geometry}


\DeclareMathOperator*{\argmax}{arg\,max}
\DeclareMathOperator*{\argmin}{arg\,min}
\newcommand{\linia}{\rule{\linewidth}{0.5pt}}

% my own titles
\makeatletter
\renewcommand{\maketitle}{
\begin{center}
\vspace{2ex}
{\huge \textsc{\@title}}
\vspace{1ex}
\\
\linia\\
\@author \hfill \@date
\vspace{4ex}
\end{center}
}
\makeatother
%%%

% custom footers and headers
\usepackage{fancyhdr}
\pagestyle{fancy}
\lhead{}
\chead{}
\rhead{}
\lfoot{}
\cfoot{--\thepage--}
\rfoot{}
\renewcommand{\headrulewidth}{0pt}
\renewcommand{\footrulewidth}{0pt}
%

% code listing settings
\usepackage{listings}
\lstset{
    language=Python,
    basicstyle=\ttfamily\small,
    aboveskip={1.0\baselineskip},
    belowskip={1.0\baselineskip},
    columns=fixed,
    extendedchars=true,
    breaklines=true,
    tabsize=4,
    prebreak=\raisebox{0ex}[0ex][0ex]{\ensuremath{\hookleftarrow}},
    frame=lines,
    showtabs=false,
    showspaces=false,
    showstringspaces=false,
    keywordstyle=\color[rgb]{0.627,0.126,0.941},
    commentstyle=\color[rgb]{0.133,0.545,0.133},
    stringstyle=\color[rgb]{01,0,0},
    numbers=left,
    numberstyle=\small,
    stepnumber=1,
    numbersep=10pt,
    captionpos=t,
    escapeinside={\%*}{*)}
}

%%%----------%%%----------%%%----------%%%----------%%%
     
\begin{document}
\title{Jetson setup notes}
\author{Topi}
\date{\today}

\maketitle

\begin{enumerate}
	\item Nvidia webpages have good setup guide. Needed to download etcher for image burning. 
	\item For X11 forwarding need to enable in client .ssh/config - ForwardX11 yes.
	\item You can check the jetpack version from file: /etc/nv\_tegra\_release - though the following script ought to take are of correct versions.
	\item Using \url{https://github.com/dusty-nv/jetson-inference/blob/master/docs/aux-docker.md}
	\begin{itemize}
	\item \texttt{git clone --recursive https://github.com/dusty-nv/jetson-inference}
	\item \texttt{cd jetson-inference}	
	\item \texttt{docker/run.sh}
	\end{itemize}
	\item Opted for building from source - cannot get image feed from container\dots
	\begin{itemize}
	\item \texttt{sudo apt update}
	\item \texttt{sudo apt-get install git cmake libpython3-dev python3-numpy}
	\item \texttt{git clone --recursive https://github.com/dusty-nv/jetson-inference}
	\item Create jetson-inference/build directory
	\item In jetson-inference/build/ run: \texttt{cmake ../}
	\item Select to install pytorch!
	\item Model downloader can be found from: \texttt{cd jetson-inference/tools} and (\texttt{./download-models.sh})
	\item Then, still in jetson-inference/build, run: \texttt{make} and \texttt{sudo make install} and \texttt{sudo ldconfig}
	\end{itemize}
	\item Now you just go to: \texttt{cd jetson-inference/build/aarch64/bin/} and run: \texttt{./imagenet.py images/orange\_0.jpg images/test/output\_0.jpg}
	\item Turns out no HQ camera support on jetson\dots
		\begin{itemize}
			\item \url{https://developer.ridgerun.com/wiki/index.php?title=Raspberry_Pi_HQ_camera_IMX477_Linux_driver_for_Jetson#Installing_the_Driver_-_Option_A:_Debian_Packages_.28Recommended.29}
			\item Seem rather impossible to resolve\dots
		\end{itemize}
	\item Create virtual env: \texttt{python3 -m venv venv}
	\item \texttt{source venv/bin/activate}
	\item Running the imagenet on rpi video: \texttt{./imagenet.py rtsp://192.168.1.72:8554/unicast file://test.mp4}
	\item Install redis server: \texttt{sudo apt install redis}
	\item It starts when booted\dots
	\item \texttt{sudeo service redis status}
	\item redis bind on localhost is mayhem! You can config it in /etc/redis/redis.conf - bind  localhost (127\dots) (comment it out - but beaware of the security risk.)
	\item Set the protected mode to no - again in the redis config file.	
\end{enumerate}





%\bibliography{refs}
%\bibliographystyle{plain}

\end{document}

